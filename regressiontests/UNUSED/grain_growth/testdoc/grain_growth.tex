\documentclass[12pt]{article}

\usepackage{times}
\usepackage{amsmath}
\usepackage{amssymb}
\usepackage{amsfonts}


\textheight 9.35in \textwidth 6.8in %
\oddsidemargin -0.15in \evensidemargin -0.15in
\topmargin -15mm
\footskip 8mm
\headheight 13pt

\begin{document}

\begin{center}
{\large {\bf Modelling Microstructure during Phase Change - Beckermann-Wang Model}}
\end{center}

\subsection*{Objective}

This testcase tests the accuracy of the Beckermann-Wang model for microstructure
formation during solidification. One-way coupling is employed and results
compared to golden output.

\subsection*{Definition}

We model the solidification of the {\em{Al-7Si}} alloy in a pseudo-1D
rod of dimension $100 \times 1 \times 1$. Material properties (conductivity, 
specific heat and density) are assumed constant in the solid and liquid phases.
Microstructure formation during solidification is implemented using the Beckermann-Wang
model. In this model dendritic crystals are assumed spherical (dendritic envelope) 
with uniform radius. This grain radius, $R$ is given by
\begin{equation}
\frac{\partial R}{\partial t} = \omega^2\left(\frac{D_lm(\kappa - 1)C_e}{\pi^2\Gamma}\right)
\end{equation}
where
$D_l$ is the diffusion in the liquid, $\kappa$ the partition coefficient, $m$ the liquidus slope,
$\Gamma$ the Gibbs-Thomson coefficient, $C_e$ the concentration of the liquid in the 
dendritic envelope and $\omega$ is the dimensionless solutal undercooling at the dendrite tip
\begin{equation}
\omega = \frac{C_e - C_l}{C_e(1-\kappa)}
\end{equation}

Temperatures calculated at the microstructure scale are not fed back to the macro-scale.
However, macro-scale temperatures are used to initialise the evolution of the microstructure
at the start of each macro-scale time step. This is a one-way coupling solution technique.

\subsection*{Metrics}

We compare the grain radius at various times at cells $1,2,3$ with golden output.
A percentage error is used for this comparison.

\subsection*{Truchas Model}

To run this example in Truchas, we select the heat conduction model
and {\em{grain\_growth\_bw}} phase-change model. The time-step grows from an
initial timestep $dt_0=0.0001$, heat conduction is implicitly solved. 
An orthogonal pseudo-1D mesh of dimensions $[0,0,0] \Rightarrow [0.1,0.01,0.012]$ 
on a $100 \times 1 \times 1$ grid is employed.
A homogenous Neumann temperature boundary condition 
is employed at both ends of the rod. The liquid initially occupies the entire
volume with initial temperature $T_0 = 670K$.  We run the simulation to $t=3.0s$.
 
\subsection*{Results}

The results of these runs are compared to a golden solution.

\end{document}