\documentclass[12pt]{article}

\usepackage{times}
\usepackage{latexmake}

\usepackage{amsmath}
\usepackage{amssymb}
\usepackage{amsfonts}


\textheight 9.35in \textwidth 6.8in %
\oddsidemargin -0.15in \evensidemargin -0.15in
\topmargin -15mm
\footskip 8mm
\headheight 13pt


\begin{document}

\begin{center}
{\large {\bf Heat Transfer Boundary Condition Test}}
\end{center}

\subsubsection*{Objective}

This test problem tests all exterior boundary conditions that are 
available in the heat transfer module, except viewfactor radiation.


\subsubsection*{Definition}

We consider the 3D conductive heat transfer problem that is defined as
follows. The problem domain is the cube $[0,11]^3$, which is
discretized using a logically rectangular hex mesh of size $11 \times
11 \times 11$. The conductivity is represented by the polynomial
$$
K(T) = 1 + \frac{T}{10} + \frac{T^2}{1000}.
$$
The boundary conditions are a constant Dirichlet boundary conditions
on the boundaries $x=0$, $y=0$, and $z=0$
\begin{eqnarray*}
T(x,y,z) &=& 10,\quad x=0 \\
T(x,y,z) &=& 20,\quad y=0 \\
T(x,y,z) &=& 30,\quad z=0.
\end{eqnarray*} 
On the boundary at $x=11$ we prescribe a radiative heat loss boundary
condition where the reference temperature is $T_0 = 0$ and emissivity
$\epsilon = 1$, such that
$$
q = - \epsilon K (T^4 - T_0^4),
$$ 
where $K$ is the conductivity and $T$ the temperature on the
boundary. 
The boundary at $y=11$, has a homogeneous Neuman boundary condition
$$
{\bf n} \cdot \nabla T (x,y,z) = 0,\quad y=11,
$$
and the boundary $z=11$ has a convective heat loss boundary condition
that is implemented using the HTC boundary condition type. The
parameters are the reference temperature $T_0 = 50$, and the heat
transfer coefficient $\beta = 50$, such that
$$
q = - \beta(T-T_0).
$$






\subsubsection*{Metrics}

We compare the temperature at the final time step to the
temperature that was obtained in a reference run (golden output). We
use the $\ell^\infty$-norm for this comparison. 

\subsubsection*{Truchas Model}

To run this example in truchas, we select the heat conduction model
exclusively. The boundary conditions are selected as described above.
The heat transfer problem is solved using Newton Krylov with flexible
GMRES as a preconditioner, which is itself preconditioned by SSOR.

We this problem for ten time steps on two different meshes. The
first mesh is an orthogonal hex mesh, while the second mesh is only
logically rectangular. It was generated from the orthogonal mesh by
randomizing the interior vertices by at most 20\% of their original
position (see the cubit journal file for the randomized mesh).

On the orthogonal mesh, we use the ortho operator and on the
randomized mesh, we use the support operator discretization.


\subsubsection*{Results}

The results of these runs are compared to a reference run (golden
output) as there is no analytic solution to this problem.

\end{document}