\documentclass[12pt]{article}

\usepackage{times}
%\usepackage{latexmake}

\usepackage{amsmath}
\usepackage{amssymb}
\usepackage{amsfonts}


\textheight 9.35in \textwidth 6.8in %
\oddsidemargin -0.15in \evensidemargin -0.15in
\topmargin -15mm
\footskip 8mm
\headheight 13pt


\begin{document}

\begin{center}
{\large {\bf Multiple Phase Change Solidification with Isothermal, Non-Isothermal, and Alloy with Scheil Model}}
\end{center}

\subsection*{Objective}

This test problem is intended to test having multiple phase changes defined and active with thermal conduction but without fluid flow or volume change.  This is testing the accurate traversal of the defined phase diagram.

\subsection*{Definition}

The problem begins with a bar of liquid at a temperature of 301.  A Dirchlet temperature boundary condition of 0 is applied to the left end and 350 is applied to the right end. 

\subsection*{Metrics}

The temperature and enthalpy of the liquid and solid portions of the bar are determined by the thermal conductivity in each phase.  The cells undergoing phase change follow the phase diagram defined below.  The volume of fluid (VOF) of the liquid should be 1 to the right of the phase change and 0 to the left.  These metrics can be approximated by:

\begin{enumerate}
\item Compare the temperature along the bar at specific time steps to the golden solution.
\item Compare the enthalpy along the bar at specific time steps to the golden solution.
\item Compare the volume of fluid along the bar at specific time steps to the golden solution.
\end{enumerate}

For each metric the error is determined by taking the L2 norm of the corresponding vectors from Truchas and the golden solution.

\subsection*{Truchas Model}

The problem domain is a bar of length 200 with height of 10 and depth of 10.  The bar is subdivided into 40 cells along the length of the bar.  The initial temperature is 301 with a concentration of 0.10.  

This first phase change is an alloy phase change using the Scheil model.  The melting temperature of the solvent is 358.6, the liquidus temperature is 300, the liquidus slope is -586, the latent heat is 500, the eutectic temperature is 250, and the partition coefficient constant is 0.15.

The second phase change is an isothermal phase change. The melting temperature is 200 and the latent heat is 2000.

The final phase change is non-isothermal.  The melting temperature is 100, the liquidus temperature is 125 and the latent heat is 1000.

The left end of the bar has a Dirichlet temperature boundary condition of 0 and the right end is 350.  The boundaries along the length have a temperature hneumann boundary condition.  The simulation is run for 500 time steps with an initial time step of 1e-6 with growth of 1.005.  The minimum time step is 1e-15 and the maximum of 25.

\subsection*{Results}

To be added.

\subsection*{Critique}

To be added.

\subsection*{References}

\end{document}