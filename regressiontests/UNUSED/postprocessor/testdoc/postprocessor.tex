\documentclass[12pt]{article}

\usepackage{times}
\usepackage{amsmath}
\usepackage{amssymb}
\usepackage{amsfonts}


\textheight 9.35in \textwidth 6.8in %
\oddsidemargin -0.15in \evensidemargin -0.15in
\topmargin -15mm
\footskip 8mm
\headheight 13pt

\begin{document}

\begin{center}
{\large {\bf Postprocessor Tests}}
\end{center}

\subsection*{Objective}

These testcases are designed to test the following capabilities of the 
postprocessor:
\begin{itemize}
\item creation of restarts 
\item creation of visualisation files
\item postprocessor diagnostics
\item recovery of linear aborts
\end{itemize}

For each postprocessor testcase, the {\em{truchas}} code is run and the output
postprocessed using pre-existing macro files. The resulting output 
from the postprocessor is then verified. The verification method used depends
on the particular type of testcase.

\subsubsection*{Restarts}

Both standard and mapped restarts are tested. Mapped restarts where
the mapping occurs from one mesh to one mesh and from two meshes to one mesh
are tested. The resulting restart file from the given restart testcase
is then verified by running a {\em{truchas}} restart simulation.

\subsubsection*{Visualisation}

The following visualisation packages are tested:
\begin{itemize}
\item GMV
\item TecPlot
\item EnSight
\item VTK
\end{itemize}
For each package binary writes are tested except for the {\em{TecPlot}}
testcase where only an ascii writer is currently supported. Verification methods
for the resulting outputted graphics file are employed using batch commands
to ensure automation. For example, {\em{gmvbatch}} is used for the {\em{GMV}} 
testcases and {\em{ens\_checker}} is used for the {\em{EnSight}} testcases.

\subsubsection*{Diagnostics}

The following postprocessor diagnostic commands are tested:
\begin{itemize}
\item query
\item stat
\item probe
\end{itemize}

Verification of the results from these commands involves simply checking
for the existence of text files that contain postprocessor output resulting 
from these commands.

\subsubsection*{Aborts}

Linear residual aborts are tested. Note that in this testcase a {\em{truchas}}
simulation that is designed to fail is run. {\em{GMV}} files are created
containing the aborted residual variables. These are verified using the
{\em{gmvbatch}} command. 

\end{document}